% !TEX root = main.tex

\section{Exercise 4}
\subsection{Continuous time state space model}
We introduce two new states (elevation $e$ and its rate $\dot{e}$) in order to be able to control them. The continuous time system can now be written as $\mathbf{\dot{x}} = \mathbf{A}_c\mathbf{x}+ \mathbf{B}_c \mathbf{u}$, with
\begin{subequations}
    \begin{align}
        \mathbf{x} &= \begin{bmatrix}
            \lambda\\
            r\\
            p\\
            \dot{p}\\
            e\\
            \dot{e}\\
        \end{bmatrix}\\
        \mathbf{u} &= \begin{bmatrix}
            p_c\\
            e_c\\
        \end{bmatrix}\\
    \mathbf{A_c} &= \begin{bmatrix}
        0 & 1 & 0          & 0          & 0          & 0         \\
        0 & 0 & -K_2       & 0          & 0          & 0         \\
        0 & 0 & 0          & 1          & 0          & 0         \\
        0 & 0 & -K_1K_{pp} & -K_1K_{pd} & 0          & 0         \\
        0 & 0 & 0          & 0          & 0          & 1         \\
        0 & 0 & 0          & 0          & -K_3K_{ep} & -K_3K_{ed}
    \end{bmatrix}\\
    \mathbf{B_c} &= \begin{bmatrix}
        0         & 0        \\
        0         & 0        \\
        0         & 0        \\
        K_1K_{pp} & 0        \\
        0         & 0        \\
        0         & K_3K_{ep}
    \end{bmatrix}
    \end{align}
\end{subequations}
\subsection{Discrete time state space model}
The system was discretized using the forward Euler method. Using \cref{eq:eulerfwd}, we get the model $\mathbf{x_{k+1}} = \mathbf{A}_d\mathbf{x_k}+\mathbf{B}_d \mathbf{u_k}$, where
\begin{subequations}
    \begin{align}
    \mathbf{A_d} &= \begin{bmatrix}
        1 & h & 0           & 0            & 0           & 0         \\
        0 & 1 & -hK_2       & 0            & 0           & 0         \\
        0 & 0 & 1           & h            & 0           & 0         \\
        0 & 0 & -hK_1K_{pp} & 1-hK_1K_{pd} & 0           & 0         \\
        0 & 0 & 0           & 0            & 1           & h         \\
        0 & 0 & 0           & 0            & -hK_3K_{ep} & 1-hK_3K_{ed}
    \end{bmatrix}\\
    \mathbf{B_d} &= \begin{bmatrix}
        0          & 0        \\
        0          & 0        \\
        0          & 0        \\
        hK_1K_{pp} & 0        \\
        0          & 0        \\
        0          & hK_3K_{ep}
    \end{bmatrix}
    \end{align}
\end{subequations}

\subsection{Objective function and inequality constraints on elevation}
In this exercise, we minimized the cost function
\begin{equation}
        \phi =\sum (\lambda_i - \lambda_f)^2 + r_{p}p_{ci}^2 + r_{e}e_{ci}^2
\end{equation}
which can be written as
\begin{subequations}
    \begin{align}
        \phi &= \sum(x^{T}\mathbf{Q}x+u^{T}\mathbf{R}u) \\
        \mathbf{Q} &= \begin{bmatrix}
        1 & 0 & 0 & 0 & 0 & 0         \\
        0 & 0 & 0 & 0 & 0 & 0         \\
        0 & 0 & 0 & 0 & 0 & 0         \\
        0 & 0 & 0 & 0 & 0 & 0         \\
        0 & 0 & 0 & 0 & 0 & 0         \\
        0 & 0 & 0 & 0 & 0 & 0         
        \end{bmatrix}\\
        \mathbf{R} &= \begin{bmatrix}
            q_1        & 0        \\
            0          & q_2        
        \end{bmatrix}
    \end{align}
\end{subequations}
Having the cost function on this form makes it very easy to change or add weighting for different states and inputs.

We restricted the helicopter to obey the non-linear inequality constraint
\begin{equation}
e_k \geq \alpha e^{-\beta(\lambda_k-\lambda_t)^{2}}
\end{equation}
every timestep $k$ until the time horizon.
Since this constraint is non-linear, a non-linear solver was needed. The matlab function \texttt{fmincon} was used to compute the optimal trajectory through the state space.

\todo[inline]{MAYBE ADD EXPLAIN FMINCON SOLUTION METHOD}

\todo[inline]{ADD PLOTS OF THE OPTIMAL PATH!!}
\todo[inline]{DISCUSS RESULTS!!}

\subsection{Experimental results}
The optimal trajectory was fed into the helicopter, both with and without feedback

\todo[inline]{ADD PLOTS OF THE EXPERIMENT PATH!!}
\todo[inline]{DISCUSS RESULTS!!}

\subsection{Decoupled states}

\todo[inline]{ADD INFO ABOUT DECOUPLED STATES}