% !TEX root = main.tex

\section{Optimal control of Pitch/Travel without Feedback}
\subsection{}
The time continuous system can be written as $\mathbf{\dot{x}} = \mathbf{A}_c\mathbf{x}+ \mathbf{B}_c\mathbf{u}$, with
\begin{subequations}
    \begin{align}
        \mathbf{x} &= \begin{bmatrix}
            \lambda\\
            r\\
            p\\
            \dot{p}
        \end{bmatrix}\\
        \mathbf{A}_c &= \begin{bmatrix}
            0 & 1 & 0 & 0\\
            0 & 0 & -K_2 & 0\\
            0 & 0 & 0 & 1\\
            0 & 0& -K_1K_{pp} & -K_1K_{pd}
        \end{bmatrix} \\
        \mathbf{B}_c &= \begin{bmatrix}
            0\\
            0\\
            0\\
            K_1K_{pp}
        \end{bmatrix}
    \end{align}
\end{subequations}

This is a model of the helicopter with pitch controller. The model inputs are set points for the pitch controller. 

\subsection{}
The system was discretized using forward Euler:
\begin{equation}
    \mathbf{\dot{x}} = \mathbf{A}_c\mathbf{x}_k + \mathbf{B}_c\mathbf{u}_k \approx \frac{\mathbf{x}_{k+1}-\mathbf{x}}{h}
\end{equation}

Rearranging, we get
\begin{equation}
    \mathbf{x}_{k+1} = (\mathbf{A}_ch + \mathbf{I})\mathbf{x}_k + h\mathbf{B}_c\mathbf{u}_k
\end{equation}

and we see that
\begin{subequations}
    \begin{align}
        \mathbf{A}_d &= \mathbf{A}_ch+\mathbf{I} \\
        \mathbf{B}_d &= h\mathbf{B}_c
    \end{align}

\end{subequations}
\todo{Insert Ad and Bd here as well}

\subsection{}
The cost function \todo{insert cost function (15)} consists of two terms, $\lambda_i - \lambda_f)^2$ and $qp_{ci}^2$. The first one aims to penalize deviation from the reference point $\lambda_f$, while the second one penalizes inputs $p_{ci}^2$. The relative cost of the two terms is set using the weight q. 

One possible downside to the cost term $(\lambda_i - \lambda_f)^2$ could be that it does not take into account that $\lambda_i = \lambda_ i + 2\pi$, resulting in very poor pathing if exposed to a large perturbation. On the other hand, it might be desirable to actually perform one or multiple revolutions.
\todo{finner vi flere ulemper?}

\subsection{}
\todo{insert code + plots}
The helicopter does not end in the desired point $x_f$ for any of the weights. This is largely \todo{er det flere grunner?} because of model inaccuracies; discretization errors from using forward Euler and linearization errors, in addition to inherent model errors/simplifications. Also, the system is affected by noise (e.g. wind currents), as well as measurement errors. This all leads to an error in $dot{\lambda}$, causing the helicopter to drift away from the reference point. \todo{mulig siste setning er overflødig.}